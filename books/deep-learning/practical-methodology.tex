
\chapter{Practical Methodology}

Successfully applying deep learning techniques requires more than a good knowledge of what algorithms exist and
the principles that explain how they work.
A good machine learning practitioner also needs to know how to choose an algorithm for a particular application and
how to monitor and respond to feedback obtained from experiments in order to improve a machine learning system.
Practitioners need to decide whether to gather more data,
increase or decrease model capacity,
add or remove regularizing features,
improve the optimization of a model,
improve approximate inference in a model,
or debug the software implementation of the model.


\begin{tcolorbox}
  \begin{itemize}
  \item a good knowledge of what algorithms exist
  \item the principles that explain how they work
  \item how to choose an algorithm for a particular application
  \item how to monitor and respond to feedback
    \begin{itemize}
    \item whether to gather more data
    \item increase or decrease model capacity
    \item add or remove regularizing features
    \item improve the optimization 
    \item improve approximate inference
    \item debug the software
    \end{itemize}
  \end{itemize}
\end{tcolorbox}


Some practical process:
\begin{enumerate}
\item Determine your goal -- what error metric to use, and your target value for this error metric.
  These goals and error metrics should be driven by the problem that the application is intended to solve.
\item Establish a working end-to-end pipeline as soon as possible, including the estimation of the appropriate performance metrics.
\item Instrument the system well to determine bottlenecks in performance.
  Diagnose which components are performing worse than expected and whether poor performance is due to overfitting, underfitting, or a defect in the data or software.
\item Repeatedly make incremental changes such as gathering new data, adjusting hyperparameters, or changing algorithms, based on specific findings from your instrumentation.
\end{enumerate}

