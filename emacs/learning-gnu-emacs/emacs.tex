\documentclass{book}
\newcommand{\head}[1]{\textnormal{\textbf{#1}}}
\newcommand{\keyboard}[1]{\textnormal{texttt{#1}}}
\usepackage[a4paper, inner=1.5cm, outer=3cm, top=2cm, bottom=3cm, bindingoffset=1cm]{geometry}
\usepackage{hyperref}
\usepackage{booktabs}
\begin{document}
\tableofcontents{}

\chapter{Emacs Basics}

\section{Introducing Emacs}

Emacs is important because of the integration of different things you need to do.
Any editor, no matter how simpler or complex, has the same basic functions.
If you can learn one, you can learn any of them.
Learning to use an editor is basically a matter of learning finger habit.
Good finger habits can make you an incredibly fast typist.
Intellectually, it's possible to absorb a lot from one reading, but you can form only a few new habit each day.
Don't feel obliged  to learn them all at once; pick something, practice it, and move on to the next topic.
Time spent developing good habits is time well spent.

\section{Understanding Files and Buffers}
You don't really edit files.
Instead, Emacs copies the content of a file into a temporary buffer and you edit that.
The file on disk doesn't change until you save the buffer.

\section{A Word About Modes}

mode: Emacs becomes sensitive to the task at hand.
Modes allows Emacs to the kind of editor you want for different tasks.
A buffer can be in only one major mode at a time.

minor modes: defines a particular aspect of Emacs's behavior and can be turned on and off within a major mode.

If you are good at Lisp programming, you can add your own modes. Emacs is almost infinitely extensible.

\section{Emacs commands}
How do you give commands?
Each command has a formal name, which is the name of a Lisp routine. Some command names are quite long. As a result, we need some way to abbreviate commands. Emacs ties a command name to a short sequence of keystrokes. This tying of commands to keystrokes is known as binding. 

The author of Emacs try to bind the most frequently used commands to the key sequences that are the easiest to reach:
\begin{itemize}
\item The most commonly used commands are bound to C-n(where n is any character).
\item Slightly less commonly used commands are bound to M-n.
\item Other commonly used commands are bound to C-x something.
\item Some specialized commands are bound to C-c something. These commands often relate to one of the more specialized modes, such as Java or HTML mode.
\item typing M-x long-command-name Enter.(This works for any command really)
\end{itemize}

\begin{verbatim}
You should define your own key bindings if you find yourself using the long form of a command all the time.
\end{verbatim}

\section{File}

\begin{tabular}{ll}
  \toprule[1.5pt]
  \head{Bounding} & \head{Function} \\
  \midrule
  C-x C-f & find-file \\
  C-x C-v & find-alternate-file \\
  C-x i & insert-file \\
  C-x C-s & save-buffer \\
  C-x C-w & write-file \\
  \bottomrule[1.5pt]
\end{tabular}

\section{Leaving Emacs}
C-x C-c

\section{Getting Help}
C-h C-h

\section{Version}
M-x version 

\section{Access Menus With Keyboard}
F10


\chapter{Editing}

Emacs offers lots of ways to move around in files.
The more ways you learn, the fewer keystrokes you'll need to get to the part of the file you want to edit.
Learning any editor is primarily a matter of forming certain finger habits rather than memorizing what the book says.
You will learn the right finger habits only if you start typing.

\section{Moving the Cursor}
It's easier to memorize commands if you remember what the letters stand for.
\subsection{Basic Cursor Motion}
\begin{tabular}{ll}
  \toprule[1.5pt]
  \head{Bounding} & \head{Function} \\
  \midrule
  C-f & forward-char \\
  C-b & backward-char \\
  M-f & forward-word \\
  M-b & backward-word \\
  \midrule
  C-n & next-line \\
  C-p      & previous-line      \\
  \midrule
  C-a      & beginning-of-line  \\
  C-e      & end-of-line        \\
  M-a      & backward-sentence  \\
  M-e      & forward-sentence   \\
  \midrule
  \verb|M-}|      & forward-paragraph  \\
  \verb|M-{|      & backward-paragraph \\
  \midrule
  C-x ]    & forward-page       \\
  C-x [    & backward-page      \\
  \bottomrule[1.5pt]
\end{tabular}

\begin{verbatim}
Ctrl command generally move in small units than their associated Meta commands.
\end{verbatim}

\section{Moving a Screen (or More) at a Time}
\begin{tabular}{ll}
  \toprule[1.5pt]
  \head{Bounding} & \head{Function} \\
  \midrule
  C-v      & scroll-up-command   \\
  M-v      & scroll-down-command \\
  M->      & end-of-buffer       \\
  M-<      & beginning-of-buffer \\
  & goto-line           \\
  & goto-char           \\
  C-M-v    & scroll-other-window \\
  \bottomrule
\end{tabular}



\section{Deleting Text}
\begin{tabular}{ll}
  \toprule[1.5pt]
  \head{Bounding} & \head{Function} \\
  \midrule
  Del      & delete-backward-char   \\
  C-d      & delete-char            \\
  \midrule
  M-Del    & backward-kill-word     \\
  M-d      & kill-word              \\
  \midrule
  C-k      & kill-line              \\
  M-k      & kill-sentence          \\
  C-x Del  & backward-kill-sentence \\
  \bottomrule
\end{tabular}

\section{The Kill Ring}
In Emacs, killing is not fatal, but in fact, quite the opposite.
Text that has been killed is not gone forever but is hidden in an area called the kill ring.
The kill ring is an internal storage area where Emacs put things you've copied or deleted. 

What exactly goes into the kill ring?
About the only thing that Emacs doesn't save in the kill ring is single characters, deleted with \keyboard{Del} or \keyboard{C-d}.

Emacs is clever about what it puts into the kill ring.
When it is assembling a big block of text from a group of deletions, it always assembles the text correctly.

Emacs stops assembling these blocks of text as soon as you give any comand that isn't a kill command.
\linebreak
\linebreak
\begin{tabular}{ll}
  \toprule[1.5pt]
  \head{Bounding} & \head{Function} \\
  \midrule
  C-y      & yank     \\
  M-y      & yank-pop \\
  \bottomrule
\end{tabular}


\section{Marking Text to Delete, Move, or Copy}
In Emacs, you select text by defining an area called a region. To define a region, you use a secondary pointer called a mark. You set the mark at one end of the region by pressing \keyboard{C-Space} or \keyboard{C-@}, then move the cursor to the other end of the region.


\begin{tabular}{ll}
  \toprule[1.5pt]
  \head{Bounding} & \head{Function} \\
  \midrule
  C-Space  & set-mark-command        \\
  C-@      & set-mark-command        \\
  C-x C-x  & exchange-point-and-mark \\
  M-h      & mark-paragraph          \\
  C-x h    & mark-whole-buffer       \\
  C-x C-p  & mark-page               \\
  \midrule
  C-w      & kill-region             \\
  M-w      & kill-ring-save          \\
  \bottomrule
\end{tabular}

    
\section{Editing Tricks and Shortcuts}
\subsection{Fixing Transpositions}
\begin{tabular}{ll}
  \toprule[1.5pt]
  \head{Bounding} & \head{Function} \\
  \midrule
  C-t      & transpose-char      \\
  M-t      & transpose-word      \\
  C-x C-t  & transpose-lines     \\
                  & transpose-sentence  \\
                  & transpose-paragraph \\
  \bottomrule
\end{tabular}



\subsection{Changing Capitalization}
\begin{tabular}{ll}
  \toprule[1.5pt]
  \head{Bounding} & \head{Function} \\
  \midrule
  M-c      & capitalize-word \\
  M-u      & upcase-word     \\
  M-l      & downcase-word   \\
  \bottomrule
\end{tabular}


\section{Undoing Changes}
\begin{tabular}{ll}
  \toprule[1.5pt]
  \head{Name} & \head{Meaning} \\
  \midrule
  revert-buffer & Replace current buffer text with the text of the visited file on disk. \\
  recover-file  & Visit file FILE, but get contents from its last auto-save file.        \\
  backup file    & the same as the name of the file, with a tilde(\~) added.                              \\
  auto-save file & the same as the name of the file, with a sharp(\#) added to the beginning and the end. \\
  \bottomrule
\end{tabular}








\end{document}
%%% Local Variables:
%%% mode: latex
%%% TeX-master: t
%%% End:
