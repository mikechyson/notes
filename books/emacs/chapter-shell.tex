
\chapter{Shell Mode}

To invoke shell mode, type \verb|M-x shell|.


\section{Common Key Bindings}

Hint: C-c is a prefix command.


\begin{description}
\item [TAB] complete the command or variable at the point 
\item [M-?] complete dynamically 
\item [C-c C-b] move the cursor to the backward command 
\item [C-c C-f] move the cursor to the forward command 
\item [C-c C-c] like C-c in Terminal 
\item [C-c C-d] like C-d in Terminal 
\item [C-c C-l] list the commands you have typed 
\item [C-c C-p] move the cursor to the previous command line 
\item [C-c C-n] move the cursor to the next command line 
\item [C-c C-o] delete the output of the current command line 
\item [C-M-l] move the cursor to previous command lien output
\item [C-c C-s] write the privous output into a file (overwrite)
\item [C-c C-u] kill the input between the first character and the current cursor, like C-u in Terminal
\item [C-c SPC] accumulate input, like $\backslash$ in terminal
\item [C-c .] insert the last previous argument
\item [M-n] next input
\item [M-p] previous input
\item [M-r] backward isearch, like C-r in Terminal
\item [C-c M-r] privious matching input
\item [C-c M-s] next matching input  
\item [C-c M-o] clear buffer, like C-l in Terminal
\end{description}



\section{Multiple Shell}

To open multiple shell, rename the current shell with \verb|M-x rename-buffer|.
After the rename, you can open a new shell with \verb|M-x shell|.


