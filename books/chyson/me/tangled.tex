
\chapter{纠结}

我是一个纠结的人。

我首先接触的操作系统是windows,后来学习编程,接触了linux操作系统,再后来又使用MacOS,我不断得在尝试,却没有一个能满足我的心意。

我不喜欢windows的不自由,所有的都是闭源的,并且使用起来不顺手。
我喜欢linux的开源与自由,期间我使用CentOS和Ubuntu等当作自己的日常操作系统使用过一段时间。
但是,它们也无端的报一些异常,甚至会崩溃。
不想再折腾了,很累,就使用了MacOS系统,Unix-like,和Linux很接近,不至于系统的崩溃,但时常会有一些小问题。


我的纠结普遍存在于我的生活中,
曾经,我使用WordPress来搭建自己的博客,
这种可以快速搭建的博客,使用的时候体验却很差,并且不易于修改和管理。
我也知道这是矛盾的两个方面,快速搭建和灵活性一般是难以同时得到满足的。

我的手机也是,我很喜欢使用小屏幕手机,iphonese那款,
但它有个致命的缺陷,手机低温关机,
在能正常使用作为通讯工具的前提下。
之后也期待这可以出一款和iphone se类似,却没有低温关机问题的。
在iphone se2出来的时候,那个外形却不是我喜欢的,
但我更讨厌刘海儿屏。


我的纠结来自于自己的无能。
我不像Jobs,在对现有的手机和电脑不满的时候,可以开发出iphone和macbook。
我也不像Donald,在对排版系统不满的时候,可以自己开发出TeX排版系统。



