\chapter{TeX Mode}
Emacs comes with a package for editing TeX and LaTeX files. However, this package is extremely limited in its functionality. A far better package called \keyword{AUC TeX} can help you write your papers dfficiently.

\section{Installation}
The modern and stronly recommended way of installing AUCTeX is by using the Emacs package integrated in Emacs and greater(ELPA). Simply do \verb|M-x list-packages RET|, mark the auctex package for installation with \verb|i| and hit \verb|x| to execute the installation procedure. That's all.

But on my CentOS 8, there is no \keyword{auctex} on MELPA repository, So I compile it from source.

\begin{enumerate}
\item Download the source file. \href{http://ftp.gnu.org/pub/gnu/auctex/auctex-12.2.zip}{auctex}
\item Unzip the source file.
\item change to the unzipped directory
\item \verb|./configure|
\item \verb|make|
\item \verb|make install|
\end{enumerate}


To uninstall AUCTeX, \verb|make uninstall|.


\begin{tcolorbox}
  Note: AUCTeX is more strict on the \LaTeX !
  There is a error saying \verb|! Undefined control sequence. l.1218 \ExplSyntaxOn|.
  I comment \keyword{ExplSyntexOn} out to solve the problem Temperately
\end{tcolorbox}


\section{Common Commands}

\begin{description}
\item[C-c \{] Make a pair of braces.
\item[C-c C-s] Insert a sectioning command.
\item[C-c C-e] Insert a environment.
\item[C-M-a] Move point to the \verb|\begin| of the current environment.
\item[C-M-e] Move point to the \verb|\end| of the current environment.
\item[M-RET] Insert an item.
\item[C-M-i] Complete TeX symbol before point.
\item[C-c C-m] Prompt for the name of a TeX macro.
\item[C-c *] Mark the current section.
\item[C-c .] Mark the current environment.
\item[C-c \%] Add or remove ‘\%’ from the beginning of each line in the current paragraph.
\item[C-c C-q C-p or M-q] Fill and indent the current paragraph.
\item[C-c C-q C-e] Fill and indent the current environment.
\item[C-c C-q C-s] Fill and indent the current section.
\end{description}

\section{Folding Macros and Environments}

A popular complaint about markup languages like TeX and LaTeX is that there is too much clutter in the source text and that one cannot focus well on the content.
With AUCTeX’s folding functionality you can collapse those items and replace them by a fixed string, the content of one of their arguments, or a mixture of both.
In order to use this feature, you have to activate \verb|TeX-fold-mode| which will activate the auto-reveal feature and the necessary commands to hide and show macros and environments.
You can activate the mode in a certain buffer by typing the command \verb|M-x TeX-fold-mode RET| or using the keyboard shortcut \verb|C-c C-o C-f|.
If you want to use it every time you edit a LaTeX document, add it to a hook:

\lstset{language=Lisp}
\begin{lstlisting}
  (add-hook 'LaTeX-mode-hook (lambda () (TeX-fold-mode 1)))
\end{lstlisting}


If it should be activated in all AUCTeX modes, use \verb|TeX-mode-hook| instead of \verb|LaTeX-mode-hook|.

Once the mode is active there are several commands available to hide and show macros, environments and comments:
\begin{description}
\item[C-c C-o C-b] Hide all foldable items in the current buffer according to the setting of \verb|TeX-fold-type-list|.
\item[C-c C-o C-r] Hide all configured macros in the marked region.
\item[C-c C-o C-p] Hide all configured macros in the pararaph containing point.
\item[C-c C-o C-e] Hide the current environment.
\item[C-c C-o C-c] Hide the comment point is located on.
\item[C-c C-o b] Permanently unfold all macros and environments in the current buffer.
\item[C-c C-o r] Permanently unfold all macros and environments in the marked region.
\item[C-c C-o p] Permanently unfold all macros and environments in the pararaph containing point.
\item[C-c C-o i] Permanently show the macros or environment on which point currently is located.
\item[C-c C-o C-o] Hide or show items according to the current context.
\end{description}

\section{Narrowing}

\begin{description}
\item[C-x n g] Make text outside current group invisible.
\item[C-x n e] Make text outside current environment invisible.
\item[C-x n w] Unnarrow.
\end{description}


\section{Documentation about Macros and Packages}
\begin{description}
\item[C-c ?] Get documentation about the packages installed on your system, using ‘texdoc’ to find the manuals. 
\end{description}


%%% Local Variables:
%%% mode: latex
%%% TeX-master: "emacs"
%%% End:
