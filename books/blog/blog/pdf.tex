
\chapter{PDF}

\section{Model}

models.py:
\lstset{language=Python}
\begin{lstlisting}
from django.db import models


# Create your models here.

class Category(models.Model):
    category_name = models.CharField('Category', max_length=200)

    def __str__(self):
        return self.category_name


class PDF(models.Model):
    name = models.CharField(max_length=200)
    author = models.CharField(max_length=200, null=True, blank=True)
    keywords = models.CharField(max_length=200, null=True, blank=True)
    date = models.DateTimeField(blank=True)
    url = models.CharField(max_length=200)
    category = models.ManyToManyField(to=Category)
    introduction = models.CharField(max_length=1000, null=True, blank=True)

    def __str__(self):
        return self.name


class Comment(models.Model):
    pdf = models.ForeignKey(PDF, on_delete=models.CASCADE)
    content = models.CharField('Content', max_length=10000)
    author = models.CharField('Author', max_length=200)
    date = models.DateTimeField('Date', auto_now_add=True)
    email = models.CharField('Email', max_length=200, null=True, blank=True)

    def __str__(self):
        return self.content
  
\end{lstlisting}

\verb|null=True| is used to allow null value in database.
\verb|blank=True| is used to allow null in admin view.

\verb|__str__| is used to show name instead of info table primary key.
This is usefull in showing foreign key.


\section{View}

views.py:
\begin{lstlisting}
from django.shortcuts import render, get_object_or_404
from django.http import HttpResponse, FileResponse, Http404
from .models import PDF, Comment
from django.template import loader
from django.views.generic import DetailView, ListView
from django.core.paginator import Paginator


# Create your views here.

def index(request):
    pdf_list = PDF.objects.order_by('-name').reverse()
    paginator = Paginator(pdf_list, 10)
    page_number = request.GET.get('page')
    page_obj = paginator.get_page(page_number)
    context = {
        'page_obj': page_obj
    }
    return render(request, 'pdf/index.html', context)


def detail(request, pk):
    pdf = get_object_or_404(PDF, pk=pk)
    comments = Comment.objects.filter(pdf=pk)[:5]
    context = {
        'pdf': pdf,
        'comments': comments
    }
    return render(request, 'pdf/detail.html', context)

\end{lstlisting}


Paginator is easy to use class to provide page function.
The corresponding \verb|index.html| is:
\lstset{language=Html}
\begin{lstlisting}

    <div class="pdf_list">
        <table>
            <tr>
                <th>PDF NAME</th>
                <th>AUTHOR</th>
                <th>KEY WORDS</th>
            </tr>
            
                <tr>
                    <th><a href="">{{ pdf.name }}</a></th>
                    <th>{{ pdf.author }}</th>
                    <th>{{ pdf.keywords }}</th>
                </tr>
            
        </table>
    </div>

    <div class="pagination">
    <span class="step-links">
        
            <a href="?page=1">&laquo; first</a>
            <a href="?page={{ page_obj.previous_page_number }}">previous</a>
        

        <span class="current">
            Page {{ page_obj.number }} of {{ page_obj.paginator.num_pages }}.
        </span>

        
            <a href="?page={{ page_obj.next_page_number }}">next</a>
            <a href="?page={{ page_obj.paginator.num_pages }}">last &raquo;</a>
        
    </span>
    </div>

    <p>No pdfs are available.</p>

\end{lstlisting}


\section{Url}

urls.py:
\lstset{language=Python}
\begin{lstlisting}
from django.urls import path

from . import views

app_name = 'pdf'
urlpatterns = [
    path('', views.index, name='index'),
    path('<int:pk>/', views.detail, name='detail')
]

\end{lstlisting}

\section{Install app in admin}

admin.py:
\begin{lstlisting}
from django.contrib import admin
from .models import PDF, Comment, Category

  
class CommentAdmin(admin.ModelAdmin):
    list_display = ('content', 'author', 'date', 'pdf')
    search_fields = ['content', 'author', 'pdf']
    list_filter = ['pdf']


class PDFAdmin(admin.ModelAdmin):
    list_display = ('name', 'author', 'date')
    list_filter = ['date']
    search_fields = ['name']


admin.site.register(Comment, CommentAdmin)
admin.site.register(PDF, PDFAdmin)
admin.site.register(Category)

\end{lstlisting}