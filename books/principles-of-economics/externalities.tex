
\chapter{Externalities}

Markets do many things well, but they do not do everything well.

The market failures fall under a general category called \keyword{externalities}.
An externality arises when a person engages in an activity that influences the well-being of a bystander and yet neither pays nor receives any compensation for that effect.
If the impact on the bystander is adverse, it is called a negative externality; if it is beneficial, it is called a positive externality.



In the presence of externalities, society’s interest in a market outcome extends beyond the well-being of buyers and sellers in the market; it also includes the well-being of bystanders who are affected. Because buyers and sellers neglect the external effects of their actions when deciding how much to demand or supply, the market equilibrium is not efficient when there are externalities. That is, the equilibrium fails to maximize the total benefit to society as a whole.



\section{Externalities and market inefficiency}

\keyword{internalizing an externality}:
altering incentives so that people take account of the external effects of their actions




Negative externalities in production or consumption lead markets to produce a larger quantity than is socially desirable. Positive externalities in production or consumption lead markets to produce a smaller quantity than is socially desirable. To remedy the problem, the government can internalize the externality by taxing goods that have negative externalities and subsidizing goods that have positive externalities.



\section{Private solutions to externalities}

\subsection{The types of private solutions}


\begin{itemize}
\item moral codes
\item social sanctions
\item charities
\item self-interest of the relevant parties
\item integrating different types of business
\item the interested parties to enter into a contact
\end{itemize}


\subsection{The coase theorem}

\keyword{Coase theorem}:
the proposition that if private parties can bargain without cost over the allocation of resources, they can solve the problem of externalities on their own.


The Coase theorem says that private economic actors can solve the problem of externalities among themselves. Whatever the initial distribution of rights, the interested parties can always reach a bargain in which everyone is better off and the outcome is efficient.



\subsection{Why private solutions do not always work}

Sometimes the interested parties fail to solve an externality problem because of \keyword{transaction costs}, the costs that parties incur in the process of agreeing to and following through on a bargain.




\section{Public policies toward externalities}

When an externality causes a market to reach an inefficient allocation of resources, the government can respond in one of two ways:
\begin{itemize}
\item Command-and-control policies regulate behavior directly. 
\item Market-based policies provide incentives so that private decisionmakers will choose to solve the problem on their own.
\end{itemize}



\subsection{Regulation}

The government can remedy an externality by making certain behaviors either required or forbidden.


\subsection{Pigovian taxes and subsidies}

\keyword{Pigovian tax}:
a tax enacted to correct the effects of a negative externality


\subsection{Tradable pollution permits}

Pollution permits, like Pigovian taxes, are now widely viewed as a cost-effective way to keep the environment clean.


\subsection{Objections to the economic analysis of pollution}

People face tradeoffs.
Certainly, clean air and clean water have value. But their value must be compared to their opportunity cost — that is, to what one must give up to obtain them.









