
\chapter{相互依存性与贸易的好处}

你每天都在想用许多素不相识的人向你提供的物品与服务。
这种相互依存之所以成为可能,
是因为人们互相交易。
那些为你提供物品和服务的人并不是出于仁慈而这样做的。
相反,人们向你和其他消费者提供他们生产的物品与服务,
是因为他们也得到了某种回报。




\section{比较优势:专业化的动力}

\subsection{绝对优势}


当比较一个人、一个企业或者一个国家与另一个人、另一个企业或另一个国家的生产率时,
经济学家用\keyword{绝对优势}这个术语。
如果生产者生产一种物品所需要的投入较少,
就可以说该生产者在生产这种物品上有绝对优势。


\subsection{机会成本和比较优势}



在描述两个生产者的机会成本时,经济学家用\keyword{比较优势}这个术语。
如果一个生产者在生产X物品时放弃了较少的其他产品,
即生产X物品的机会成本较小,
我们就可以说,他在生产该物品上具有比较优势。


今天一个人有可能在两种物品的生产上都有绝对优势,
但一个人却不可能在两种物品的生产上都具有比较优势。
因为一种物品的机会成本是另一种物品机会成本的倒数,
(除非两个人有相同的机会成本)。
比较优势反应了相对的机会成本。


\subsection{比较优势与贸易}

专业化和贸易的好处不是基于绝对优势,
而是基于比较优势。
当每个人专门生产自己比较优势的物品时,
经济的总产量就增加了,
经济蛋糕的变大可用于改善每个人的状况。
人们通过以低于自己生产某种物品的机会成本的价格得到该物品而从贸易中获益。



贸易可以使社会上每个人都获益,
因为它使人们可以专门从事他们具有比较优势的互动。


\subsection{贸易的价格}

对从贸易中获益的双方而言,
他们进行贸易的价格在两种机会成本之间。


\subsection{比较优势的应用}


一位经济学家认为,你不应该仅仅因为你比你的配偶更擅长洗碗,就总是负责洗碗。
(比较优势)


