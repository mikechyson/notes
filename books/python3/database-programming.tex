
\chapter{Database Programming}

There are two commonly used database:
\begin{enumerate}
\item RDBMS (Relational Database Management System).
  These systems use tables (spreadsheet-like grids) with rows equating to records and columns equating to fields.
  The tables and the data they hold are created and manipulated using statements written in SQL (Structured Query Language).
\item DBM (Database Manager).
  It stores any number of key-value items.
\end{enumerate}


\section{DBM databases}

The \verb|shelve| module provides a wrapper around a DBM that allows us to interact with the DBM as though it were a dictionary, providing that we use only string keys and pickable values.
Behind the scenes the \verb|shelve| module converts the keys and values to and from \verb|bytes| objects.


\url{https://github.com/mikechyson/python3/blob/master/c12_database/dvds_dbm.py}


\section{SQL databases}

To make it as easy as possible to switch between database backends, PEP 249 (Python Database API Specification v2.0) provides an API specification called DB-API 2.0 that database interfaces ought to honor.
There are two major objects specified by the API, the connection object and the cursor object.


\begin{table}[!ht]
  \centering
  \begin{tabular}{lp{0.8\columnwidth}}
    \toprule{}
    \head{Syntax} & \head{Description} \\
    \midrule
    db.close() & Closes the connection. \\
    db.commit() & Commits any pending transaction to the database; does nothing for databases that don't support transactions. \\
    db.cursor() & Returns a databse cursor object through which queries can be executed. \\
    db.rollback() & Rolls back any pending transaction to the state that existed before the transaction began; does nothing for databases that don't support transactions. \\
    \bottomrule
  \end{tabular}
  \caption{DB-API 2.0 Connection Object Methods}
\end{table}


\begin{table}[!ht]
  \centering
  \begin{tabular}{p{0.2\columnwidth}p{0.7\columnwidth}}
    \toprule
    \head{Syntax} & \head{Description} \\
    \midrule
    c.arraysize & The (readable/writable) number of rows that \verb|fetchall()| will return if no size is specified \\
    c.fetchmany(size) & Returns a sequence of rows (each row it self being a sequence); \verb|size| default to \verb|c.arraysize| \\
    c.fetchall() & Returns a sequence of all the rows that have not yet been fetched \\
    c.fetchone() & Returns the next row of the query result set as a sequence, or \verb|None| when the results are exhausted. \\
    c.description & A read-only sequence of 7-tuples (\verb|name|, \verb|type_code|, \verb|display_size|, \verb|internal_size|, \verb|precision|, \verb|scale|, \verb|null_ok|), describing each successive column of cursor \verb|c| \\
    c.execute(sql, params) & Executes the SQL query in string \verb|sql|, replacing each palceholder with the corresponding parameter from the \verb|params| sequence of mapping if given \\
    c.execute(sql, seqofparams) & Executes the SQL query once for each item in the \verb|seq_of_params| sequence of sequences or mappings; this method should not be used for operations that create result sets (such as \verb|SELECT| statements) \\
    c.close() & Closes the cursor, \verb|c|; this is done automatically when the curosr goes out of scope
  \end{tabular}
  \caption{DB-API 2.0 Cursor Object Attributes and Methods}
\end{table}


\url{https://github.com/mikechyson/python3/blob/master/c12_database/dvds_sql.py}


