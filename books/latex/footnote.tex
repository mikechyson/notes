\chapter{Footnotes}

\begin{lstlisting}
  \footnote{hello world}
  \section[title without footnote]{This is a section\protect\footnote{section footnote}}

  \footnote[number]{text}

  \footnotemark[number] % produces a superscripted number in the text as a
  % footnote mark. If the optional argument wasn't given, it's also stepping and using
  % the internal footnote counter. No footnote will be generated.

  \footnotetext[number]{text} % generates a footnote without putting a
  % footnote mark into the text without stepping the internal footnote counter.

  \footnoterule % used to alter the footnote line

  \renewcommand{\footnoterule}{\noindent\smash{\rule[3pt]{\textwidth}{0.4pt}}}
  % \rule[raising]{width}{height} draws a line, here 0.4 pt thick, and as wide as the text, raised a bit by 3 pt.
  % \smash , let the line pretend to have a height and a depth of zero, so it's occupying no vertical space at all.
\end{lstlisting}

Example:
\begin{lstlisting}
  Hello World\footnote{hello world}
\end{lstlisting}
\begin{tcolorbox}
Hello World\footnote{hello world}.  
\end{tcolorbox}



