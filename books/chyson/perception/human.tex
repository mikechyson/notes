
\chapter{做人}


\section{自律}
自律真的是挺难的,那自律的难点在哪里呢?

因为需要自律的时候往往伴随着痛苦,当我们想到之后的痛苦,我们就不愿意付诸行动。

比如说健身,拉伸的时候,特别的痛,那我们就不愿意去拉伸。
做力量的训练的时候,特别的累,特别的难受,那我们就不愿意去做力量训练。
同样的,对于核心的训练,也特别的累。
矛盾的是,拉伸如果不痛,就起不到拉伸的效果,
力量训练的时候,如果肌肉如果不酸,也起不到力量训练的效果,
同样的,腹部如果不酸,也起不到核心训练的效果。

比如说学习,就算一个人喜欢学习,学习的是自己感兴趣的学科,其中难免也有困难的部分,
比如公式的记忆,难点的理解等,
在一个知识点百思不得其解的时候,那是一种精神上的巨大痛苦。

不需要自律的往往是简单的,不需要付出努力就很容易达到的,
比如睡懒觉,玩游戏,看电视。

我们为什么要自律呢?

为了自律的结果,自律的结果是使我们更优秀,以达到我们想要的结果。
比如健身,我们可以拥有健康的身体,好看的身材。
比如学习,我们可以增加自己的知识,开阔自己的视野,增加自己的能力。

那如何更好的自律呢?
\begin{itemize}
\item 拥有一个简单的开始,比如穿好了健身设备,准备好了学习的书本等。
\item 少去想过程中的痛苦,多去想自律的结果。
\end{itemize}


\section{责任}

如果人来是从最初的单细胞生物进化而来的,
那单细胞生物赖以生存下来的趋利避害的特性也被我们继承了下来,
就像我们恐高一样,因为我们知道,如果掉下去,我们很大概率会死亡。

对于不利于我们生存的,我们的第一反应是躲避,
所以对于担责任,我们的第一反应是躲避的,
我们从封建社会走来,古代的我们,犯了错误往往付出生命的惨重代价,
即使是现在,担责任往往也伴随着不利的影响和麻烦。

但渐渐的,我发现,逃避责任比之担责任带来的不利影响更大,
逃避责任使我们规避了近期的微小的不利影响,
但却使我们在更长远的将来承受巨大的损失,
这些损失包括,珍贵的人际关系,健康的心理状态。


自己的错误,自己不去承担责任,那必然有其他人来承担责任,
而这些责任却不是他们的,他们自然不愿意承认,却为你背了锅,
那承担责任的人下次会远离你的,来避免类似的事情的发生,
你们的关系就会慢慢疏远。

逃避责任是一种不健康的心理状态,
这种心理状态的不利于你的成功,
你无法从错误中学习到应该的教训,
而这些教训是无价的,
再多的哲理,如果不是自己切身体会的,也起不到什么多大作用,
我们知道的责任其实已经太多太多的,
但按照这些来做的又有多少人呢?
我们都知道自律是好的,
但能做到的又有多少人呢?
我们都知道冲动不好,
但又有多少人能在极端的情况下控制好自己的情绪呢?

承担责任带来的好处是长远的,
我们要勇于承担自己应有的责任,
当然对于不是自己的责任,
我们可以选择承担或者不承担。

\section{借口}

借口其实是一种自我价值的保护,避免借口的最好方法就是自信。


我们学习不好或者考试不好的时候,
我们可以说我们没有好好学,或者考试紧张没有考好,
来保护我们的学习能力。
我们跑步速度不好的时候,
我们可以说,因为交通灯红灯比价多,或者膝盖受伤了,
来保护我们的运动能力。

其实真的没有必要,
因为我自信,
我的学习能力我自己清楚,
我的运动能力我自己清楚,
无需去为一件事情的失败去找借口,
自己心理清楚失败的原因就可以了,
这一切的一切都不会因为他人的误解而让我产生自我否定,
用下次的成功这个事实来说话就可以了。


能力分为两种:
\begin{itemize}
\item 天生的
\item 后天的
\end{itemize}

其实对我们生活产生最大影响的其实是后天能力,
既然是这样的,
何必去为自己的失败找借口呢,
我们当时有,那没必要去找借口,
因为它不会因为他人的误解而消失,
就算我们当时没有,
我们更没有必须去找借口,
知耻而后勇,
我们以后培养起来就可以了。



\section{信仰}

\begin{center}
\begin{verbatim}
Prefer to pursue the emptiness, also cannot have no pursuit.
\end{verbatim}
\end{center}

我们

我们必须有自己的信仰,

如果你内心中深信一件事有着 “崇高的目的 “,那你多半会对相伴的苦工夫视而不见,反而会认为每一滴汗水都是值得的;这像极了锻炼身体时,即使再苦再累,你也因为相信它的益处而保持心情愉悦。



\section{负面能量与自信}

这个世界上形形色色的思想形成了形形色色的人,笼统的可以将思想分成三类:
\begin{itemize}
\item 正向的
\item 中性的
\item 负向的
\end{itemize}


我们周围时常存在这这三种类型的思想,
我们无法选择身边的人,
但我们可以选择尽量的少接触此类思想,
因为思想是可以传播的。


我们要在我们的思想周围建立起保护膜,
它将思想的负面部分进行去除,
就是我们要忽略掉思想中的负面部分。
自信是一层强有力的保护膜,
它可以过滤掉对你否定的部分,
当然,我们不要太自以为是,
我们也有缺点,
我们也要听取建议,
我们需要区分建议与否定,
建议我们接受,
否定我们去除。


你要无条件的保持自信,
即使在你一无所有的时候。




\section{努力}

为什么努力?

努力不是为了给别人看的,
努力也不是用来感动自己的,
努力是为了达到自己的\keyword{目标}而进行的脑力和体力的付出。

明白了努力的目的,
我们也就没必要去假装努力来获得他人感动,或者理解,或者赞美,
我们也就没有必要去低效率的堆积时间来感动自己,
因为结果不会陪着你撒谎。




\section{后悔}

12月12日我要去参加北京二环挑战赛,
而我现在身处西安,
因为项目进度的问题,
我无法申请回北京,
因而也无法报销来回的路费,
综合算下来,
如果去,来回的路费1千多,
如果不去,报名费399元不予以退回,
我买了机票,
然后就有些许的后悔情绪伴随着我,
直到某个时刻,
我在心里坦然接受了,
感觉轻松了许多,
没有再患得患失了。



我们生活中会作出许多的选择,
有些选择我们会产生后悔的情绪,
后悔情绪会严重耗费我们的精力,
我们的决定或许是错误的,
我们的决定或许不是时间最优和金钱最优的,
但只要是自己做的决定,
并且出自自己意愿,
那么就接受自己的选择,
坦然些,
我知道我会吃亏,
但是我不在乎,
生命中,
不止存在金钱,
还有独一无二的自己情感。







\section{不争辩}


生活中,不要和愚蠢的人争辩。


同事说iphone12系列支持空间音效,
一直想用airpod pro来听听空间音效,
但他没有iphone,
然后用我的来听,
我在手机界面中把spatial audio打开了,
然后用music放了一首歌,
我再对比关掉spatial audio的音效,
我说没有区别,
他说怎么能没有区别,
说我耳朵有问题把,
让他来听,
我让他来听打开和关闭spatial audio情况的同一首歌,
他说怎么没区别,
打开之后更带劲了,
还说我耳朵是不是有问题,
我说确实听不多区别,
两个人还各执己见了一会。

而后的今天,
我再去看spatial audio的说明,
上面明确写着仅对部分视频有效果,
比如杜比视界,
现在想想真的好蠢,
何必去和愚蠢的人去争辩,
和愚蠢的人去争辩,
你也就变成了愚蠢的人,
原来我自己也是个愚蠢的人,
时常想去争辩,
证明自己是对的,
他人的理解出了问题,
排除科学上的争辩,
生活中真的没必要去亢奋的争辩,
我以后改掉这个毛病,
不要去做生活中无谓的争辩,
在确实需要一个结果的时候,
去考究,
在考究自己错误的情况下,
自己改正,
在考究他人错误的情况情况下,
自己知道就可以了。



\section{门槛}


知识的难易,技术的难易,形成不同的门槛。
而你的工资是和你的不可替代性相关的。


如果你的职业或者爱好不需要门槛或者需要很低的门槛,
那么你就需要注意了,
你的不可替代性就很低,
你容易被其他的人或者机器所代替。

高速收费员,餐厅服务员,快递分拣员,司机等等,
这些职业如今被越来越多的机器所替代,
从今而后也会有越来越多的人可能被机器所替代,
所以,在你踏入一个职业的时候,
不要被思维上的门槛高而吓到,
思维上门槛高意味着你更不容易被替代,
思想上的门槛高意味着需要更多的思维活动,
而不仅仅是体力活动。







