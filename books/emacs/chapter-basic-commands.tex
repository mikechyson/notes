\chapter{Commands}

\section{Basic Commands}
\begin{description}
\item[C-x C-f] Open or create a file.
\item[C-x i] Insert a file's content into the current cursor.
\item[C-x C-s] Save the current buffer.
\item[C-x C-w] Write current buffer into the file.
\item[C-x C-c] Exit Emacs. 
\item[C-/] Undo the previous command.
\item[C-q] useful for inserting control character 
\item[M-n] \verb|n| stands for numbers 
\item[C-u] Begin a numberic argument. 
\item[M--] Begin a negative numeric argument. 
\item[C-l] Move current buffer line to the specified window line.
\item[C-g] Quit.
\item[C-x Esc Esc] Edit and re-evaluate last complex command from last.
\item[C-x z] Repeat most recently executed command.
\item[M-/] Expand previous word ``dynamically''.
\item[M-m] Move point to the first non-whitespace character on this line.
\item[C-M-i] Complete the current symbol.
\end{description}





\section{Movement}
\begin{description}
\item[C-f] forward-char
\item[C-b] backward-char
\item[M-f] forward-word
\item[M-b] backward-word
\item[C-n] next-line
\item[C-p] previous-line
\item[C-a] Move to the beginning of the current line.
\item[C-e] Move to the end of the current line.
\item[M-a] backward-sentence
\item[M-e] forward-sentence
\item[M-\}] forward-paragraph 
\item[M-\{] backward-paragraph
\item[C-x ]] forward-page 
\item[C-x [] backward-page
\item[C-v] Move down one screen.
\item[M-v] Move up one screen.
\item[M->] Move to the end of the buffer.
\item[M-<] Move to the beginning of the buffer.
\item[M-g M-g] Go to the specified line.
\item[C-M-v] Scroll other window one screen.
\end{description}

\section{Deleting Text}

\begin{description}
\item[Del] Delete backward one character.
\item[C-d] Delete one character.
\item[M-Del] Delete backward one word.
\item[M-d] Delete one word.
\item[C-k] Kill one line from the current position.
\item[M-k] Kill one sentence.
\item[C-x Del] Delete one sentence backward.
\end{description}


\section{Yank}

In Emacs, killing is not fatal, but in fact, quite the opposite.
Text that has been killed is not gone forever but is hidden in an area called the kill ring.
The kill ring is an internal storage area where Emacs put things you've copied or deleted. 


\begin{description}
\item[C-y] Yank the most recently delete text.
\item[M-y] Yank pop.
\end{description}

\section{Marking Text}
\begin{description}
\item[C-SPC] Set mark.
\item[C-@] Set mark.
\item[C-x C-x] exchange point and mark.
\item[M-h] Mark the current paragraph.
\item[C-x h] Mark the whole buffer.
\item[C-w] Kill the marked region.
\item[M-w] Copy the marked region.
\end{description}


\section{Editing}

\begin{description}
\item[C-t] Transpose character.
\item[M-t] Transpose word.
\item[C-x C-t] Transpose lines.
\item[M-c] Capitilize the word from current position.
\item[M-u] Upcase the word from current position.
\item[M-l] Downcase the word from current position.
\end{description}


\section{Search}

\begin{description}
\item[C-s] Search incrementally.
\item[C-r] Search incrementally backward.
\item[Enter] exit incremental search.
\item[C-s C-w] Search the word under the cursor.
\item[C-s C-s] Repeat search forward.
\item[C-r C-r] Repeat search backward.
\item[M-\%] Query and replace.
\item[C-M-s] Search regular expression forward.
\item[C-M-r] Search regular expression backward.
\item[C-M-\%] Search regular expression replace
\end{description}


