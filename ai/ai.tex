\documentclass[a4paper,openright]{book}
\usepackage{xeCJK}              % cheniese japanese kerean
\usepackage{fancyhdr}
\usepackage{indentfirst}
\usepackage[onehalfspacing]{setspace}


\fancyhf{}
\fancyhead[LE]{\leftmark}
\fancyfoot[RO]{\nouppercase{\rightmark}}
\fancyfoot[LE,RO]{\thepage}
\pagestyle{fancy}
\usepackage{hyperref}
\setcounter{tocdepth}{3}
\renewcommand{\contentsname}{目录}
\renewcommand{\listfigurename}{图列表}
\renewcommand{\listtablename}{表格列表}
\renewcommand{\figurename}{图}
\renewcommand{\tablename}{表}
% \renewcommand{\partname}{部分}
% \renewcommand{\chaptername}{章节}
\renewcommand{\appendixname}{附录}
\renewcommand{\familydefault}{\ttdefault}





\begin{document}
\author{Mike Chyson}
\title{AI}
\maketitle
\tableofcontents{}
\chapter*{前言}
\addcontentsline{toc}{chapter}{前言}

最好的学习编程的方式是去写代码,最好的构建知识网络的的方式是去梳理学过知识。

我们每天都在学习新的知识,太多的知识如果不加以梳理就比较容易忘记,同时大脑也比较容易产生混乱的感觉。

鸟哥在学习linux的时候,记录自己的点点滴滴,最终形成了自己的《鸟哥的Linux私房菜》。鸟哥从不认为自己很聪明,就是认为自己不够聪明,才将点点滴滴记录,以防忘记。我们可以缓慢的行走,但只要不停下脚步,就会属于自己的精彩人生。


\part{准备}
\chapter{工具}
\section{Jupyter\protect\footnote{https://jupyter.org}}

Jupyter notebook 和Jupyter lab的优点:
\begin{description}
\item[方便的交互模式] 可以执行一个代码块,显示一段代码的运行结果。
\item[可以选择性执行代码] 可以选择性的执行某个代码快。
\item[代码帮助与查看] np.ones?查看帮助文档,np.ones??查看源码。
\end{description}



因此,jupyter很适合:
\begin{itemize}
\item 模型的开发阶段,因为这个阶段需要不断的修改代码,查看模型的效果。
\item 代码的演示,因为方便的交互性。
\item 数据分析,因为方便的交互性。
\end{itemize}


\section{Python\protect\footnote{https://www.python.org}}

在工程上更方便,包括模块测试,模块引用,等等。

python script更适合:
\begin{itemize}
\item 大型的项目。
\item 模型确定之后的模型的训练与发布。
\end{itemize}

\chapter{资源}
谷歌的引导你学习机器学习的资源网站:
https://tensorflow.google.cn/resources/learn-ml

计算机视觉方面的最新论文:
https://arxiv.org/list/cs.CV/recent


神经网络与进化计算:
https://arxiv.org/list/cs.NE/recent


人工智能的最新论文:
https://arxiv.org/list/cs.AI/recent

data science:
https://towardsdatascience.com/machine-learning/home


\appendix{}
\cleardoublepage{}
\addtocontents{toc}{\bigskip{}}
\addcontentsline{toc}{part}{Appendix}
\chapter{术语}
\chapter{符号}
\end{document}