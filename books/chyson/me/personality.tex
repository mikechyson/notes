
\chapter{我}

\section{纠结}


我是一个纠结的人。

我首先接触的操作系统是windows,后来学习编程,接触了linux操作系统,再后来又使用MacOS,我不断得在尝试,却没有一个能满足我的心意。

我不喜欢windows的不自由,所有的都是闭源的,并且使用起来不顺手。
我喜欢linux的开源与自由,期间我使用CentOS和Ubuntu等当作自己的日常操作系统使用过一段时间。
但是,它们也无端的报一些异常,甚至会崩溃。
不想再折腾了,很累,就使用了MacOS系统,Unix-like,和Linux很接近,不至于系统的崩溃,但时常会有一些小问题。


我的纠结普遍存在于我的生活中,
曾经,我使用WordPress来搭建自己的博客,
这种可以快速搭建的博客,使用的时候体验却很差,并且不易于修改和管理。
我也知道这是矛盾的两个方面,快速搭建和灵活性一般是难以同时得到满足的。

我的手机也是,我很喜欢使用小屏幕手机,iphonese那款,
但它有个致命的缺陷,手机低温关机,
在能正常使用作为通讯工具的前提下。
之后也期待这可以出一款和iphone se类似,却没有低温关机问题的。
在iphone se2出来的时候,那个外形却不是我喜欢的,
但我更讨厌刘海儿屏。


我的纠结来自于自己的无能。
我不像Jobs,在对现有的手机和电脑不满的时候,可以开发出iphone和macbook。
我也不像Donald,在对排版系统不满的时候,可以自己开发出TeX排版系统。





\section{无言}

小学的时候有门课程叫做“自然科学”,
这是我很喜欢的一门学科,
在学习电流的时候,
老师说,电子移动的方向和电流移动的方向相反,
电线导电其实是电子在金属中的运动,
我很好奇,
既然电线导电是因为电子的运动,
为什么电子运动方向不是电流的方向,
反而是和电流的方向相反呢?

这个疑问多年以后我才明白,
当初发现电流的时候是在电解液中发现的,
电解液中有游离的正负离子,
当时规定,
正离子移动的方向为电流的方向,
后来才发现,
其实用电子的流动方向为电流的方向更合理,
但鉴于历史的原因,
一直未做修正。


\section{倔强}

我的语文老师是我邻居,
语文老师的家在我家的前面,
有一次,
语文老师去我家串门,
问我作业写好了没有,
我说没有,
语文老师开玩笑的说:''作业都没有写好,别上学了。''
然后我说我不上学了,
怎么也不去上学,
说我打我,
我抱着桌子腿,
就是不去上学,
后来我都忘记我怎么去的学校,
当我站在教室门口的时候,
还听到语文老师在说我的事,
在全本学生的注视下,
我回到了自己的座位上。


\section{傻}

小时候看《射雕英雄传》,
感觉里面的梅超风好厉害,
会九阴白骨爪,
最重要的是,
她还是个瞎子,
我才知道瞎子原来可以这么厉害,
放学后,
当我出了校门,
我就闭上了眼睛,
假装自己是个瞎子,
走路的时候,
所有的感觉都变了,
自己才走几步就感觉走了长了,
偶尔还偷偷眯个眼睛,
来瞟一眼,
原来还没有到沟里,
举着双手,
小心地迈着步子,
再加上偷偷的瞟一眼,
我忘记到家的时候用了多长时间。


\section{天分}

小学没有学习,
升初中的时候我落榜了,
我以为我没有学可以上了,
最后实验中学因为招生人数不足,
将我招到了学校的3班,
一个最差的班级,
而我在班级中还是倒数的,
我用了一年的时间,
考到了班级第一,
年级前五。

大学的时候,
我和王师去图书馆,
他在看一本物理书籍,
其中一个插图配的公式他看不懂,
问了一下我,
我看了插图和公式,
告诉他这是立体角的公式,
他还嘲笑我说;”还立体角“,
因为那个时候我们还没有接触到立体角,
我们我们知道,
确实是立体角。


高三放假了而我没有回家,
在学校里里闲来无事,
我找了一本漫画书,
因为想尝试新鲜事物,
我用左手在黑板上画满了一幅动漫图,
当我画完站在最后一排看的时候,
感觉太美了,
都不舍得擦,
到周一的时候,
老师一进教室就看到了黑板上满幅的漫画,
训斥我后还让我擦掉了。




\section{一事无成}

对自己的天分我从未怀疑后,
然而我却开始思考自己为什么一事无成。


我终于找到了原因:”聚集“与”坚持“。


散:
\begin{itemize}
\item “完美世界国际版”,还研究在元神不够的情况下怎么加点,技能每升一级,攻击力增加多少,pk的时候怎么搭配技能,什么顺序做任务效率最高等
\item “穿越火线”,苦练过瞬狙,甩狙,盲狙和跳箱子
\item “英雄联盟”,主E的疾风剑豪
\item 学习围棋,研究定式,活死棋等
\item 一对一,学习了两年多的古琴,曾经左手无名指跪指和大拇指按指,有着明显的茧子
\item 一对一学习了一年多的素描,还有几周的油画,素描书籍都买了一大堆来研究
\item 练习书法,钢笔的楷书和行书,软头笔的楷书和行书,轻重缓急
\item 踢足球,院部奖
\item 打篮球,主要练习三分球,小勾手,跳投
\item 一对一,学习了一年多的钢琴
\item 买了证券考试书籍,研究证券,却没有去考证券从业资格中
\item 研究操作系统,考了红帽认证,研究编译,指令集实现,安装archlinux,gentoo,甚至打印了LFS来安装系统
\item 研究音乐理论,看了三本乐理原版书籍
\item 写过自己的玄幻小说,构思个各种异能,武魂,武技,属性,魔法等等,却发现空有构思,没有文笔,自己读的索然无味
\item C,Java,Scala,Python,Scheme,Common Lisp,Elisp,HTML,CSS,JS,LaTeX,Shell,然而现在用的却只有python
\end{itemize}



不再去分散太多的精力去研究其他没必要的东西,
现在需要做的事情有以下几点:
\begin{itemize}
\item 算法
\item 健身
\item 钢琴
\item 油画
\item 金融
\end{itemize}



\section{全程马拉松}

2020年11月29日5时43分,我从西安碑林区南稍门地铁站附近开始跑,终点为秦始皇陵兵马俑,全程42.40千米,这是我第一次跑全程马拉松。

在此之前,我跑过多次半程马拉松,却从未跑过全程,
本来是计划跑33千米的,因为12月12日有一个33千米的挑战赛,
我不知道能不能坚持跑下来,所以打算尝试下33千米,
当时也定下了兵马俑为终点,如果能坚持下来就坚持,如果不能坚持下来就努力跑33千米,
然而,信念或者目标的力量真的挺神奇的,
在我跑过半马的那刻,就努力是尝试33千米,
在跑过33千米的那刻,我把每一公里的增加都当作对自己的肯定,
就这样,一次性的从半程马拉松增加到了全程马拉松。

它的作用有:
\begin{itemize}
\item 要有目标和信念。
\item 这是一种肯定。
\end{itemize}

\section{京城二环冬季挑战赛}

2020年12月12日8时,从德胜门出发,绕京城二环一圈,全程33千米。

这是我参加的第一个线下赛,
然而和我想象的出处太大了,
参赛的人真的是太少了。


感谢宗敏同学起早全程陪跑,甚至骑行的时候还摔倒了一次。





