\chapter{Python IDE}
After adding the MELPA into your \verb|.emacs|, install the following packages.

\section{elpy}
The \keyword{elpy} package (Emacs Lisp Python Environment) provides a near-complete set of Python IDE features, including:
\begin{itemize}
\item Automatic indentation
\item Syntax highlighting
\item Auto completion
\item Syntax checking
\item Python REPL integration
\item Virtual environment support
\end{itemize}

To install and enable elpy, you add the package to your Emacs configuration.
\lstset{language=Lisp}
\begin{lstlisting}
;; install packages
(defvar my-packages
  '(better-defaults			; set up some better Emacs defaults
    material-theme			; Theme
    elpy				; Emacs Lisp Python Environment
    )
  )

;; enable elpy
(elpy-enable)
\end{lstlisting}

You can configurate the \verb|elpy| with \verb|M-x elpy-config|.

\begin{tcolorbox}
  If there is a bug: \verb|melpa.org/packages/company-20200725.2348.tar not found|.
  Try to run the command \verb|package-refresh-contents| to refresh your package database.
\end{tcolorbox}

\section{Syntax Checking}
By default, \verb|elpy| uses a syntax-checking package called \verb|flymake|.
While \verb|flymake| is built into Emacs, it only has native support four languages, and it requires significant effort to be able to support new languages.

The syntax-checking package \verb|flycheck| supports real-time syntax checking in over 50 languages and is designed to be quickly configured for new languages.
\lstset{language=Lisp}
\begin{lstlisting}
;; install packages
(defvar my-packages
  '(better-defaults			; set up some better Emacs defaults
    material-theme			; Theme
    elpy				; Emacs Lisp Python Environment
    flycheck				; on the fly syntax checking
    )
  )

;; enable flycheck
(when (require 'flycheck nil t)
  (setq elpy-modules (delq 'elpy-module-flymake elpy-modules))
  (add-hook 'elpy-mode-hook 'flycheck-mode))
\end{lstlisting}


\section{Code Fomatting}

\begin{lstlisting}
;; install packages
(defvar my-packages
  '(better-defaults			; set up some better Emacs defaults
    material-theme			; Theme
    elpy				; Emacs Lisp Python Environment
    flycheck				; on the fly syntax checking
    py-autopep8				; run autopep8 on save
    blacken				; black formatting on save
    )
  )

;; enable autopep8
(require 'py-autopep8)
(add-hook 'elpy-mode-hook 'py-autopep8-enable-on-save)

\end{lstlisting}

At this end, when you save your code Emacs will format your code automately.

\section{Python Interpreter}




\section{Itegration With Jypyter and IPython}
\begin{lstlisting}
;; Use IPython for REPL
(setq python-shell-interpreter "jupyter"
      python-shell-interpreter-args "console --simple-prompt"
      python-shell-prompt-detect-failure-warning nil)
(add-to-list 'python-shell-completion-native-disabled-interpreters
             "jupyter")
\end{lstlisting}

This will update Emacs to use IPython rather than the standard Python REPL.


The \verb|ein| package enables IPython Notebook in Emacs.
\begin{lstlisting}
  
;; install packages
(defvar my-packages
  '(better-defaults			; set up some better Emacs defaults
    material-theme			; Theme
    elpy				; Emacs Lisp Python Environment
    flycheck				; on the fly syntax checking
    py-autopep8				; run autopep8 on save
    blacken				; black formatting on save
    ein					; Emacs IPython Notebook
    )
  )

\end{lstlisting}

To start the server, use the command \verb|M-x ein:jupyter-server-start|.

\section{Debugging}
The built-in \keyword{python-mode} allows you to use Emacs for Python code debugging with \verb|pdb|.

\begin{tcolorbox}
  While the \verb|pdb| executable may exist in some Python distribution, it doesn't exist in all of them.
  You can set the variable \verb|gud-pdb-command-name| to define the command used to launch pdb.
  Add \verb|(setq gud-pdb-command-name "python -m pdb")| to \verb|.emacs| file.
\end{tcolorbox}

Use \verb|M-x pdb| to start the Python debugger.


You can step through code in \verb|pdb| using one of two keys:
\begin{description}
\item [s] steps into other functions.
\item [n] steps over other functions.
\end{description}


\section{Git Support}
In Emacs, source control support is provided by the \verb|magit| package.

\begin{lstlisting}
  
;; install packages
(defvar my-packages
  '(better-defaults			; set up some better Emacs defaults
    material-theme			; Theme
    elpy				; Emacs Lisp Python Environment
    flycheck				; on the fly syntax checking
    py-autopep8				; run autopep8 on save
    blacken				; black formatting on save
    ein					; Emacs IPython Notebook
    magit				; Git integration
    )
  )
\end{lstlisting}

\verb|M-x magit-status|.
