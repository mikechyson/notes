\chapter{Math}

\section{Basic Formula}

\begin{lstlisting}
  \section*{Quadratic equations}
  \begin{equation}
    \label{quad}
    ax^2 + bx + c = 0,
  \end{equation}
  where \( a, b\) and \( c \) are constants and \( a \neq 0 \),
  has two solutions for the variable \( x \):
  \begin{equation}
    \label{root}
    x_{1,2} = \frac{-b \pm \sqrt{b^2 -4ac}}{2a}.
  \end{equation}
  If the \emph{discriminant} \( \Detla \) with
  \[ \Delta = b^2 - 4ac \]
  is zero, then the equation (\ref{quad}) has a double solution:
  (\ref{root}) becomes
  \[ x = - \frac{b}{2a}. \]
\end{lstlisting}

% \begin{tcolorbox}
%   \section*{Quadratic equations}
%   \begin{equation}
%     \label{quad}
%     ax^2 + bx + c = 0,
%   \end{equation}
%   where \( a, b\) and \( c \) are constants and \( a \neq 0 \),
%   has two solutions for the variable \( x \):
%   \begin{equation}
%     \label{root}
%     x_{1,2} = \frac{-b \pm \sqrt{b^2 -4ac}}{2a}.
%   \end{equation}
%   If the \emph{discriminant} \( \Detla \) with
%   \[ \Delta = b^2 - 4ac \]
%   is zero, then the equation (\ref{quad}) has a double solution:
%   (\ref{root}) becomes
%   \[ x = - \frac{b}{2a}. \]

% \end{tcolorbox}


\section{Expressions within Text}
LaTeX provides the math environment in-text formulas:

\begin{lstlisting}
  \begin{math}
    expression
  \end{math}
\end{lstlisting}

LaTeX offers an alias that's doing the same:

\begin{lstlisting}
  \( expression \)
\end{lstlisting}

A third way is by using a shortcut, coming from TeX:

\begin{lstlisting}
  $expression$
\end{lstlisting}

For example:
\begin{tcolorbox}
  \begin{lstlisting}
    This is an equation: $x^2 + x = 10$
  \end{lstlisting}
  This is an equation: $x^2 + x = 10$
\end{tcolorbox}

\section{Displaying Formula}

\begin{lstlisting}
  \begin{displaymath}
    expression                  % displayed formula, centered
  \end{displaymath}
\end{lstlisting}


There are shortcuts:
\begin{lstlisting}
  \[
    expression
  \]
\end{lstlisting}

\begin{lstlisting}
  $$
  expression
  $$
\end{lstlisting}



For example:
\begin{tcolorbox}
  \begin{lstlisting}
    \begin{displaymath}
      x^2 + x = 10
    \end{displaymath}
  \end{lstlisting}
  \begin{displaymath}
    x^2 + x = 10
  \end{displaymath}

\end{tcolorbox}


\section{Numbering Equations}
\begin{tcolorbox}
  \begin{lstlisting}
    \begin{equation}
      \label{newton}
      F = ma^2
    \end{equation}
    Newton's law: \eqref{newton}.
  \end{lstlisting}
  \begin{equation}
    \label{newton}
    F = ma^2
  \end{equation}
  Newton's law: \eqref{newton}.

\end{tcolorbox}


