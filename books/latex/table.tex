\chapter{Tables}

\begin{lstlisting}
  \newcommand{\head}[1]{\textnormal{\textbf{#1}}}
  \begin{tabular}{ccc}
    \hline
    \head{Command} & \head{Declaration} & \head{Output} \\
    \hline
    \verb|\textrm| & \verb|\rmfamily| & \rmfamily Example text \\
    \verb|\textsf| & \verb|\sffamily| & \sffamily Example text \\
    \verb|\texttt| & \verb|\ttfamily| & \ttfamily Example text \\
    \hline
  \end{tabular}
\end{lstlisting}

\begin{tcolorbox}
  \begin{tabular}{ccc}
    \hline
    \head{Command} & \head{Declaration} & \head{Output} \\
    \hline
    \verb|\textrm| & \verb|\rmfamily| & \rmfamily Example text \\
    \verb|\textsf| & \verb|\sffamily| & \sffamily Example text \\
    \verb|\texttt| & \verb|\ttfamily| & \ttfamily Example text \\
    \hline
  \end{tabular}
\end{tcolorbox}


\newpage
\begin{lstlisting}

  \usepackage{booktabs} % toprule, midrule, bottomrule

  \begin{tabular}{ccc}
    \toprule[1.5pt] % British typesetters call a line a rule
    \head{Command} & \head{Declaration}& \head{Output}\\
    \midrule %
    \verb|\textrm| & \verb|\rmfamily| & \rmfamily Example text \\
    \verb|\textsf| & \verb|\sffamily| & \sffamily Example text \\
    \verb|\texttt| & \verb|\ttfamily| & \ttfamily Example text \\
    \bottomrule[1.5pt] %
  \end{tabular}

\end{lstlisting}

\begin{tcolorbox}
  
  \begin{tabular}{ccc}
    \toprule[1.5pt] % British typesetters call a line a rule
    \head{Command} & \head{Declaration}& \head{Output}\\
    \midrule %
    \verb|\textrm| & \verb|\rmfamily| & \rmfamily Example text \\
    \verb|\textsf| & \verb|\sffamily| & \sffamily Example text \\
    \verb|\texttt| & \verb|\ttfamily| & \ttfamily Example text \\
    \bottomrule[1.5pt] %
  \end{tabular}

\end{tcolorbox}


\begin{tcolorbox}
  To avoid the table exceed out the page:
\begin{verbatim}
    \resizebox{\textwidth}{!}{
      ...
    }
  
\end{verbatim}


\end{tcolorbox}


To wrap automatically in cell, use the \verb|p{width}| parameter.
For example:
\begin{lstlisting}
  
\begin{table}[htb!]
  \centering
  \begin{tabular}{p{0.3\columnwidth}p{0.3\columnwidth}p{0.3\columnwidth}}
    \toprule{}
    & \head{advantage} & \head{disadvantage} \\
    \midrule
    multiple processes & each process runs independently & communication and data sharing can be inconvenient \\
    multiple threads & can communicate simply by data sharing & more complex than single-threaded program\\
    \bottomrule
  \end{tabular}
  \caption{multiple processes and multiple threads}
\end{table}
\end{lstlisting}


\begin{table}[htb!]
  \centering
  \begin{tabular}{p{0.3\columnwidth}p{0.3\columnwidth}p{0.3\columnwidth}}
    \toprule{}
    & \head{advantage} & \head{disadvantage} \\
    \midrule
    multiple processes & each process runs independently & communication and data sharing can be inconvenient \\
    multiple threads & can communicate simply by data sharing & more complex than single-threaded program\\
    \bottomrule
  \end{tabular}
  \caption{multiple processes and multiple threads}
\end{table}





