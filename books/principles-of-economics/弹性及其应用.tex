
\chapter{弹性及其应用}

弹性衡量买者与卖者对市场条件变化的反应程度。

\section{需求弹性}

\subsection{需求价格弹性及其决定因素}

\keyword{需求价格弹性}衡量需求量对价格变动的反应程度。
如果一种物品的需求量对价格变动的反应很大,
就说明这种物品的需求量是\keyword{富有弹性}的。
如果一种物品的需求量对价格变动的反应很小,
就说明这种物品的需求是\keyword{缺乏弹性}的。


由于需求反映了形成消费者偏好的许多经济、社会与心理因素,
所有没有一个决定需求曲线弹性的简单而普遍的规律。
但根据经验,我们可以总结出某些决定需求价格弹性的经验法则。

\begin{description}
\item[相似替代品的可获得性] 有相似替代品的物品的需求往往较富有弹性。
\item[必需品与奢侈品] 必需品的需求往往缺乏弹性,而奢侈品的需求往往富有弹性。
\item[市场的定义] 任何一个市场上的需求弹性都取决于我们如何划定市场的边界。狭窄定义的市场的需求弹性往往大于宽泛定义的市场的需求弹性,因为狭窄定义的市场长的物品更容易找到相近的替代品。
\item[时间范围] 物品的需求往往在长期内更富有弹性。
\end{description}



\subsection{需求价格弹性的计算}

\begin{equation}
  \text{需求价格弹性} = \frac{\text{需求量变动百分比}}{\text{价格变动百分比}}
\end{equation}



\subsection{中点法:一个计算变动百分比和弹性的更好方法}

计算$(Q_1,P_1)$和$(Q_2,P_2)$两点间需求价格弹性的中点法可以用以下公式表示:
\begin{equation}
  \text{需求价格弹性} =
  \frac{
    \frac{Q_2-Q_1}{(Q_2+Q_1)/2}
  }{
    \frac{P_2-P_1}{(P_2+P_1)/2}
  }
\end{equation}


\subsection{各种需求曲线}

经济学家根据需求弹性对需求曲线进行分类。
当弹性大于1,需求是富有弹性的。
当弹性小于1,需求是缺乏弹性的。
当弹性等于1,需求是单位弹性的。



通过某一点的需求曲线越平坦,
需求价格弹性就越大;
通过某一点的需求曲线越陡峭,
需求价格弹性就越小。



\subsection{总收益与需求价格弹性}

