\chapter{学习}

\section{质量互变}

量变引起质变。 
你一天可以学很多知识,但变成自己的也就一部分。
你一天可以想很多东西,但付诸行动的也就一部分。
当我们想一蹴而就的时候,焦虑也就随之而来。
我们要充分考虑事物的时间属性,一幅油画的是颜料在时间空间上的积累,一首乐曲的形成是震动在时间上的积累。
当我们想去做一件事情,请将事情在时间纬度上进行分解,这样,事情更容易成功。
同样的,请在量上进行积累,只有量变才能引起质变,比如,算法题的积累,健身的积累等。



\section{应用}

我们学习很多的知识,我们可以做很多笔记,但如果离开了笔记,我们一无所知,那这些知识还不是自己的。

曾看到一片短文:
\begin{verbatim}
一个学者在外地考察了许多知识,做了一大袋的笔记,
然而在经过沙漠的上时候,被强盗抢劫了,
学者哭诉着乞求将自己的半生的心血得到的知识留给自己,
强盗说:“你的知识?你的知识是别人拿不走的,别人可以拿走的就不是你的知识。”
\end{verbatim}

应用可以加深你对知识的记忆以及理解。

\section{自学}

我们接受义务教育,然后教育结束之后,又有多少的知识是我们保留,保留的知识中,又有多少知识是我们用的到的呢?
教育,不仅仅是教给我们知识,更重要的是教给我们如果学习,在没有他人教的情况进行高效的自学。

比如,初中的数学知识,其实没记住多少,
但我数学老师讲课的方式,给我了深深的印象,
先在也在用她的教学方式去学习:
\begin{itemize}
\item 为什么引出这个?
\item 它的定义是什么?
\item 它的性质是什么?
\item 它的推论是什么?
\item 在定于,性质,推论明白的情况,我们如何灵活组合应用?
\item 不适用的场景是什么?
\item 它和什么类型,相似点和区别点是什么?
\end{itemize}


我们有需求的时候,我们可以主动的通过各种方式去满足我们的需求,这就是一种自学。
自学是我们在自己需求明确的情况,以自己的思考方式,去获取知识来解决问题的一种能力。







\section{知识}
什么是知识?

知识是一种标准,它是超脱于现实的,它即可以符合某种现实,也可以不符合另一种现实。

我们可以说$1+1=2$,也可以说$1+1=3$;
我们可以说''过直线外一点,有且只有一条直线和已知直线平行'',也可以说''过直线外一点,没有直线和已知直线平行'';
我们可以说''物体的质量是固定不变的'',也可以说''物体的质量随着速度而改变''\dots



我们感觉有的知识难,有的知识简单,
知识的难易来源于对知识的理解,
每个人对知识的理解是不同的,导致相同的知识点对不同的人难易程度是不同的。

随着知识的积累,我们形成了自己的价值观。
随着知识的积累,我们提高了对特定知识理解能力的加强。
生活中存在的太多的知识,我们可以选择我们想要接触的知识,
因为接触知识的不同,人与人之间有了观念的不同。

知识的不对等是我们致富的法宝。


\section{本质}

事物的本质都很简单,复杂的是应用,
如果我们感觉到某个知识点很难,
那么可能这个设计有不合理之处,
或者我们没有get the point,
也或者是我们考虑了非主要因素而本末倒置。




